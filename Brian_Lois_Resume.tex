%%%%%%%%%%%%%%%%%%%%%%%%%%%%%%%%%%%%%%%%%
% Medium Length Professional CV
% LaTeX Template
% Version 2.0 (8/5/13)
%
% This template has been downloaded from:
% http://www.LaTeXTemplates.com
%
% Original author:
% Trey Hunner (http://www.treyhunner.com/)
%
% Important note:
% This template requires the resume.cls file to be in the same directory as the
% .tex file. The resume.cls file provides the resume style used for structuring the
% document.
%
%%%%%%%%%%%%%%%%%%%%%%%%%%%%%%%%%%%%%%%%%

%----------------------------------------------------------------------------------------
%	PACKAGES AND OTHER DOCUMENT CONFIGURATIONS
%----------------------------------------------------------------------------------------

\documentclass{resume} % Use the custom resume.cls style

\usepackage[left=0.5in,top=0.4in,right=0.5in,bottom=0.6in]{geometry} % Document margins

\name{Brian Lois} % Your name
%\address{499 Carver Hall \\ Ames, Iowa 50011} % Your address
\address{1090 W Exchange Pkwy\#7201 \\ Allen, TX 75013} % Your secondary address (optional)
\address{(920)-660-8830 \\ BTLois@gmail.com} % Your phone number and email

\begin{document}
%\begin{center}
%{ \bf Objective:} Seeking employment as a data scientist in the Dallas area.
%\end{center}


%----------------------------------------------------------------------------------------
%	EDUCATION SECTION
%----------------------------------------------------------------------------------------

\begin{rSection}{Education}
	
	{\bf Iowa State University}, Ames, IA \hfill {\em August 2010 - May 2015} \\ 
	{\bf Ph.D. } Applied Mathematics and Electrical Engineering \hfill {\em GPA: 4.0} \\
	Research Areas: {\bf Big Data Algorithms}, {\bf Machine Learning}, {\bf Linear Algebra} \vspace{.1in}\\
	{\bf Marquette University}, Milwaukee, WI \hfill {\em August 2006 - May 2010 }\\
	B.S. Mathematics \hfill {\em GPA: 4.0}
	
\end{rSection}


%----------------------------------------------------------------------------------------
%	WORK EXPERIENCE SECTION
%----------------------------------------------------------------------------------------

\begin{rSection}{Work Experience}

\begin{rSubsection}{CapitalOne Financial}{September 2016 - Present}{Data Scientist Manager}
	\item Develop data science tools and capabilities in {\bf Python}.
	\item Review code for credit decisioning models.
\end{rSubsection}
	
\begin{rSubsection}{AT\&T }{June 2015 - September 2016}{Data Scientist}{Plano, TX}
	\item Work on the Corporate Insights team which uses big data techniques to drive efficiency within the organization.
	\item Write code in {\bf Python}, {\bf R}, and {\bf Pig}.
	\item Use {\bf natural language processing} and topic clustering analysis with Latent Dirichlet Allocation and Non-negative Matrix Factorization.
	\item Build {\bf time-series forecasting } models to determine if internal advertisements are effective.
	\item Perform social network analysis on workplace collaboration data.
	\item Design and implement novel mathematical and statistical models to predict when mechanical equipment will require maintenance.
\end{rSubsection}

\begin{rSubsection}{General Dynamics}{August 2014 - April 2015}{Statistician Practicum Student}{Des Moines, IA}
\item Member of the health analytics team.
\item Perform data analytics using SQL and SAS on the Chronic Conditions Warehouse dataset which contains all of the national Medicare claims and beneficiary data.
\end{rSubsection}

%------------------------------------------------

\begin{rSubsection}{Statistical and Applied Mathematical Sciences Institute}{July 2013}{Workshop Participant}{Raleigh, NC}
\item Worked with a diverse team of applied mathematicians and statisticians on a health care data analytics and visualization problem presented by SAS Inc.
\end{rSubsection}

%------------------------------------------------

\begin{rSubsection}{Iowa State University}{August 2010 - May 2015}{Research and Teaching Assistant}{Ames, IA}
\item Proved the first correctness result for a recursive robust {\bf principal components analysis} algorithm.  
\item Received the Graduate Teaching Excellence Award.
%\item In collaboration with colleagues, write and submit research papers to conferences and journals.
\item Mentor a group of undergraduate REU students working on a research project entitled ``Weighted Compressed Sensing."
\item Teach stand alone sections of Calculus 1, 2, and 3. 
Grade and hold office hours for various graduate and undergraduate mathematics and electrical engineering courses. 
\end{rSubsection}

\begin{rSubesection}{Marquette University}{August 2007 - May 2010}

\end{rSection}





\vfill
\begin{center}
	(over)
\end{center}
\newpage

%----------------------------------------------------------------------------------------
%	TECHNICAL STRENGTHS SECTION
%----------------------------------------------------------------------------------------

%\begin{rSection}{Technical Strengths}
%	
%	\begin{tabular}{ @{} >{\bfseries}l @{\hspace{6ex}} l }
%		Programming Languages: & R, Python, Pig, SAS, MATLAB\\
%		Hortonworks Training: & Hadoop Essentials \\
%		& Data Science \\
%		& Spark \\
%		Selected Graduate Coursework: & Design and Analysis of Algorithms \\
%		& Convex and Numerical Optimization \\
%		& Applied Linear Algebra \\
%		& Statistical Signal Processing
%
%	\end{tabular}
	
\end{rSection}
%----------------------------------------------------------------------------------------
%	AWARDS
%----------------------------------------------------------------------------------------

\begin{rSection}{Awards}
\begin{rSubsection}{Graduate College Research Excellence Award}{2015}{}{}
	\item Awarded by the university ``to recognize the best of the best graduating students.''
	%``to recognize graduate students for outstanding research accomplishments'' and 
\end{rSubsection}
 
\begin{rSubsection}{Lambert Research Fellow}{January 2014 - May 2014}{}{}
	\item Awarded to one or two upper-level mathematics graduate students each year.
\end{rSubsection} 

\begin{rSubsection}{Teaching Excellence Award}{2013}{}{}
	\item Awarded to no more than 10\% of teaching assistants.
\end{rSubsection}

\begin{rSubsection}{Department of Mathematics Outstanding Senior Award}{2010}{}{}
\item Awarded to one graduating senior each year.
\end{rSubsection}
\end{rSection}







%----------------------------------------------------------------------------------------
%	EXAMPLE SECTION
%----------------------------------------------------------------------------------------

\begin{rSection}{Leadership and Service Activities}

\begin{rSubsection}{Graduate and Professional Student Senate}{August 2013 - November 2015}{Math Department Representative}{  }
\item Authored, passed, and assisted in implementation of a major bill to reform and update the way travel funding is distributed.
\item Initiated and passed significant changes to the fiscal year 2015 budget.
\end{rSubsection}

\begin{rSubsection}{Mathematics Graduate Student Organization}{August 2013 - May 2014}{Executive Council Member}{ }
\item Organized a career day where graduates of the program return to talk about their current careers.
\item Secured funding from both the GPSS and the math department. 
\end{rSubsection}

\begin{rSubsection}{Information Processing for Big Data symposium}{2014}{Reviewer}{}
\item Peer-review papers submitted to the IEEE Global Signal and Information Processing Conference.
\end{rSubsection}

\end{rSection}

\begin{rSection}{Selected Publications}
\underline{A Correctness Result for Online Robust PCA}\\
{\bf Brian Lois} and Namrata Vaswani\\
Submitted to : Information Theory, IEEE Transactions on, March 2015 (under review)\\
Available: arXiv:1409.3959


\underline{Recursive Robust PCA or Recursive Sparse Recovery in Large but Structured Noise}\\
Chenlu Qiu, Namrata Vaswani, {\bf Brian Lois}, Leslie Hogben\\
Information Theory, IEEE Transactions on (Volume: 60 , Issue: 8) Aug 2014

\underline{Performance Guarantees for Undersampled Recursive Sparse Recovery in Large but Structured Noise} \\
{\bf Brian Lois}, Namrata Vaswani, Chenlu Qiu \\
IEEE Global Conference on Signal and Information Processing, Dec 2013
\end{rSection}





%----------------------------------------------------------------------------------------

\end{document}
